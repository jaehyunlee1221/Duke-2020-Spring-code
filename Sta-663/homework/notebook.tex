
% Default to the notebook output style

    


% Inherit from the specified cell style.




    
\documentclass[11pt]{article}

    
    
    \usepackage[T1]{fontenc}
    % Nicer default font (+ math font) than Computer Modern for most use cases
    \usepackage{mathpazo}

    % Basic figure setup, for now with no caption control since it's done
    % automatically by Pandoc (which extracts ![](path) syntax from Markdown).
    \usepackage{graphicx}
    % We will generate all images so they have a width \maxwidth. This means
    % that they will get their normal width if they fit onto the page, but
    % are scaled down if they would overflow the margins.
    \makeatletter
    \def\maxwidth{\ifdim\Gin@nat@width>\linewidth\linewidth
    \else\Gin@nat@width\fi}
    \makeatother
    \let\Oldincludegraphics\includegraphics
    % Set max figure width to be 80% of text width, for now hardcoded.
    \renewcommand{\includegraphics}[1]{\Oldincludegraphics[width=.8\maxwidth]{#1}}
    % Ensure that by default, figures have no caption (until we provide a
    % proper Figure object with a Caption API and a way to capture that
    % in the conversion process - todo).
    \usepackage{caption}
    \DeclareCaptionLabelFormat{nolabel}{}
    \captionsetup{labelformat=nolabel}

    \usepackage{adjustbox} % Used to constrain images to a maximum size 
    \usepackage{xcolor} % Allow colors to be defined
    \usepackage{enumerate} % Needed for markdown enumerations to work
    \usepackage{geometry} % Used to adjust the document margins
    \usepackage{amsmath} % Equations
    \usepackage{amssymb} % Equations
    \usepackage{textcomp} % defines textquotesingle
    % Hack from http://tex.stackexchange.com/a/47451/13684:
    \AtBeginDocument{%
        \def\PYZsq{\textquotesingle}% Upright quotes in Pygmentized code
    }
    \usepackage{upquote} % Upright quotes for verbatim code
    \usepackage{eurosym} % defines \euro
    \usepackage[mathletters]{ucs} % Extended unicode (utf-8) support
    \usepackage[utf8x]{inputenc} % Allow utf-8 characters in the tex document
    \usepackage{fancyvrb} % verbatim replacement that allows latex
    \usepackage{grffile} % extends the file name processing of package graphics 
                         % to support a larger range 
    % The hyperref package gives us a pdf with properly built
    % internal navigation ('pdf bookmarks' for the table of contents,
    % internal cross-reference links, web links for URLs, etc.)
    \usepackage{hyperref}
    \usepackage{longtable} % longtable support required by pandoc >1.10
    \usepackage{booktabs}  % table support for pandoc > 1.12.2
    \usepackage[inline]{enumitem} % IRkernel/repr support (it uses the enumerate* environment)
    \usepackage[normalem]{ulem} % ulem is needed to support strikethroughs (\sout)
                                % normalem makes italics be italics, not underlines
    

    
    
    % Colors for the hyperref package
    \definecolor{urlcolor}{rgb}{0,.145,.698}
    \definecolor{linkcolor}{rgb}{.71,0.21,0.01}
    \definecolor{citecolor}{rgb}{.12,.54,.11}

    % ANSI colors
    \definecolor{ansi-black}{HTML}{3E424D}
    \definecolor{ansi-black-intense}{HTML}{282C36}
    \definecolor{ansi-red}{HTML}{E75C58}
    \definecolor{ansi-red-intense}{HTML}{B22B31}
    \definecolor{ansi-green}{HTML}{00A250}
    \definecolor{ansi-green-intense}{HTML}{007427}
    \definecolor{ansi-yellow}{HTML}{DDB62B}
    \definecolor{ansi-yellow-intense}{HTML}{B27D12}
    \definecolor{ansi-blue}{HTML}{208FFB}
    \definecolor{ansi-blue-intense}{HTML}{0065CA}
    \definecolor{ansi-magenta}{HTML}{D160C4}
    \definecolor{ansi-magenta-intense}{HTML}{A03196}
    \definecolor{ansi-cyan}{HTML}{60C6C8}
    \definecolor{ansi-cyan-intense}{HTML}{258F8F}
    \definecolor{ansi-white}{HTML}{C5C1B4}
    \definecolor{ansi-white-intense}{HTML}{A1A6B2}

    % commands and environments needed by pandoc snippets
    % extracted from the output of `pandoc -s`
    \providecommand{\tightlist}{%
      \setlength{\itemsep}{0pt}\setlength{\parskip}{0pt}}
    \DefineVerbatimEnvironment{Highlighting}{Verbatim}{commandchars=\\\{\}}
    % Add ',fontsize=\small' for more characters per line
    \newenvironment{Shaded}{}{}
    \newcommand{\KeywordTok}[1]{\textcolor[rgb]{0.00,0.44,0.13}{\textbf{{#1}}}}
    \newcommand{\DataTypeTok}[1]{\textcolor[rgb]{0.56,0.13,0.00}{{#1}}}
    \newcommand{\DecValTok}[1]{\textcolor[rgb]{0.25,0.63,0.44}{{#1}}}
    \newcommand{\BaseNTok}[1]{\textcolor[rgb]{0.25,0.63,0.44}{{#1}}}
    \newcommand{\FloatTok}[1]{\textcolor[rgb]{0.25,0.63,0.44}{{#1}}}
    \newcommand{\CharTok}[1]{\textcolor[rgb]{0.25,0.44,0.63}{{#1}}}
    \newcommand{\StringTok}[1]{\textcolor[rgb]{0.25,0.44,0.63}{{#1}}}
    \newcommand{\CommentTok}[1]{\textcolor[rgb]{0.38,0.63,0.69}{\textit{{#1}}}}
    \newcommand{\OtherTok}[1]{\textcolor[rgb]{0.00,0.44,0.13}{{#1}}}
    \newcommand{\AlertTok}[1]{\textcolor[rgb]{1.00,0.00,0.00}{\textbf{{#1}}}}
    \newcommand{\FunctionTok}[1]{\textcolor[rgb]{0.02,0.16,0.49}{{#1}}}
    \newcommand{\RegionMarkerTok}[1]{{#1}}
    \newcommand{\ErrorTok}[1]{\textcolor[rgb]{1.00,0.00,0.00}{\textbf{{#1}}}}
    \newcommand{\NormalTok}[1]{{#1}}
    
    % Additional commands for more recent versions of Pandoc
    \newcommand{\ConstantTok}[1]{\textcolor[rgb]{0.53,0.00,0.00}{{#1}}}
    \newcommand{\SpecialCharTok}[1]{\textcolor[rgb]{0.25,0.44,0.63}{{#1}}}
    \newcommand{\VerbatimStringTok}[1]{\textcolor[rgb]{0.25,0.44,0.63}{{#1}}}
    \newcommand{\SpecialStringTok}[1]{\textcolor[rgb]{0.73,0.40,0.53}{{#1}}}
    \newcommand{\ImportTok}[1]{{#1}}
    \newcommand{\DocumentationTok}[1]{\textcolor[rgb]{0.73,0.13,0.13}{\textit{{#1}}}}
    \newcommand{\AnnotationTok}[1]{\textcolor[rgb]{0.38,0.63,0.69}{\textbf{\textit{{#1}}}}}
    \newcommand{\CommentVarTok}[1]{\textcolor[rgb]{0.38,0.63,0.69}{\textbf{\textit{{#1}}}}}
    \newcommand{\VariableTok}[1]{\textcolor[rgb]{0.10,0.09,0.49}{{#1}}}
    \newcommand{\ControlFlowTok}[1]{\textcolor[rgb]{0.00,0.44,0.13}{\textbf{{#1}}}}
    \newcommand{\OperatorTok}[1]{\textcolor[rgb]{0.40,0.40,0.40}{{#1}}}
    \newcommand{\BuiltInTok}[1]{{#1}}
    \newcommand{\ExtensionTok}[1]{{#1}}
    \newcommand{\PreprocessorTok}[1]{\textcolor[rgb]{0.74,0.48,0.00}{{#1}}}
    \newcommand{\AttributeTok}[1]{\textcolor[rgb]{0.49,0.56,0.16}{{#1}}}
    \newcommand{\InformationTok}[1]{\textcolor[rgb]{0.38,0.63,0.69}{\textbf{\textit{{#1}}}}}
    \newcommand{\WarningTok}[1]{\textcolor[rgb]{0.38,0.63,0.69}{\textbf{\textit{{#1}}}}}
    
    
    % Define a nice break command that doesn't care if a line doesn't already
    % exist.
    \def\br{\hspace*{\fill} \\* }
    % Math Jax compatability definitions
    \def\gt{>}
    \def\lt{<}
    % Document parameters
    \title{Homework05-Copy1}
    
    
    

    % Pygments definitions
    
\makeatletter
\def\PY@reset{\let\PY@it=\relax \let\PY@bf=\relax%
    \let\PY@ul=\relax \let\PY@tc=\relax%
    \let\PY@bc=\relax \let\PY@ff=\relax}
\def\PY@tok#1{\csname PY@tok@#1\endcsname}
\def\PY@toks#1+{\ifx\relax#1\empty\else%
    \PY@tok{#1}\expandafter\PY@toks\fi}
\def\PY@do#1{\PY@bc{\PY@tc{\PY@ul{%
    \PY@it{\PY@bf{\PY@ff{#1}}}}}}}
\def\PY#1#2{\PY@reset\PY@toks#1+\relax+\PY@do{#2}}

\expandafter\def\csname PY@tok@w\endcsname{\def\PY@tc##1{\textcolor[rgb]{0.73,0.73,0.73}{##1}}}
\expandafter\def\csname PY@tok@c\endcsname{\let\PY@it=\textit\def\PY@tc##1{\textcolor[rgb]{0.25,0.50,0.50}{##1}}}
\expandafter\def\csname PY@tok@cp\endcsname{\def\PY@tc##1{\textcolor[rgb]{0.74,0.48,0.00}{##1}}}
\expandafter\def\csname PY@tok@k\endcsname{\let\PY@bf=\textbf\def\PY@tc##1{\textcolor[rgb]{0.00,0.50,0.00}{##1}}}
\expandafter\def\csname PY@tok@kp\endcsname{\def\PY@tc##1{\textcolor[rgb]{0.00,0.50,0.00}{##1}}}
\expandafter\def\csname PY@tok@kt\endcsname{\def\PY@tc##1{\textcolor[rgb]{0.69,0.00,0.25}{##1}}}
\expandafter\def\csname PY@tok@o\endcsname{\def\PY@tc##1{\textcolor[rgb]{0.40,0.40,0.40}{##1}}}
\expandafter\def\csname PY@tok@ow\endcsname{\let\PY@bf=\textbf\def\PY@tc##1{\textcolor[rgb]{0.67,0.13,1.00}{##1}}}
\expandafter\def\csname PY@tok@nb\endcsname{\def\PY@tc##1{\textcolor[rgb]{0.00,0.50,0.00}{##1}}}
\expandafter\def\csname PY@tok@nf\endcsname{\def\PY@tc##1{\textcolor[rgb]{0.00,0.00,1.00}{##1}}}
\expandafter\def\csname PY@tok@nc\endcsname{\let\PY@bf=\textbf\def\PY@tc##1{\textcolor[rgb]{0.00,0.00,1.00}{##1}}}
\expandafter\def\csname PY@tok@nn\endcsname{\let\PY@bf=\textbf\def\PY@tc##1{\textcolor[rgb]{0.00,0.00,1.00}{##1}}}
\expandafter\def\csname PY@tok@ne\endcsname{\let\PY@bf=\textbf\def\PY@tc##1{\textcolor[rgb]{0.82,0.25,0.23}{##1}}}
\expandafter\def\csname PY@tok@nv\endcsname{\def\PY@tc##1{\textcolor[rgb]{0.10,0.09,0.49}{##1}}}
\expandafter\def\csname PY@tok@no\endcsname{\def\PY@tc##1{\textcolor[rgb]{0.53,0.00,0.00}{##1}}}
\expandafter\def\csname PY@tok@nl\endcsname{\def\PY@tc##1{\textcolor[rgb]{0.63,0.63,0.00}{##1}}}
\expandafter\def\csname PY@tok@ni\endcsname{\let\PY@bf=\textbf\def\PY@tc##1{\textcolor[rgb]{0.60,0.60,0.60}{##1}}}
\expandafter\def\csname PY@tok@na\endcsname{\def\PY@tc##1{\textcolor[rgb]{0.49,0.56,0.16}{##1}}}
\expandafter\def\csname PY@tok@nt\endcsname{\let\PY@bf=\textbf\def\PY@tc##1{\textcolor[rgb]{0.00,0.50,0.00}{##1}}}
\expandafter\def\csname PY@tok@nd\endcsname{\def\PY@tc##1{\textcolor[rgb]{0.67,0.13,1.00}{##1}}}
\expandafter\def\csname PY@tok@s\endcsname{\def\PY@tc##1{\textcolor[rgb]{0.73,0.13,0.13}{##1}}}
\expandafter\def\csname PY@tok@sd\endcsname{\let\PY@it=\textit\def\PY@tc##1{\textcolor[rgb]{0.73,0.13,0.13}{##1}}}
\expandafter\def\csname PY@tok@si\endcsname{\let\PY@bf=\textbf\def\PY@tc##1{\textcolor[rgb]{0.73,0.40,0.53}{##1}}}
\expandafter\def\csname PY@tok@se\endcsname{\let\PY@bf=\textbf\def\PY@tc##1{\textcolor[rgb]{0.73,0.40,0.13}{##1}}}
\expandafter\def\csname PY@tok@sr\endcsname{\def\PY@tc##1{\textcolor[rgb]{0.73,0.40,0.53}{##1}}}
\expandafter\def\csname PY@tok@ss\endcsname{\def\PY@tc##1{\textcolor[rgb]{0.10,0.09,0.49}{##1}}}
\expandafter\def\csname PY@tok@sx\endcsname{\def\PY@tc##1{\textcolor[rgb]{0.00,0.50,0.00}{##1}}}
\expandafter\def\csname PY@tok@m\endcsname{\def\PY@tc##1{\textcolor[rgb]{0.40,0.40,0.40}{##1}}}
\expandafter\def\csname PY@tok@gh\endcsname{\let\PY@bf=\textbf\def\PY@tc##1{\textcolor[rgb]{0.00,0.00,0.50}{##1}}}
\expandafter\def\csname PY@tok@gu\endcsname{\let\PY@bf=\textbf\def\PY@tc##1{\textcolor[rgb]{0.50,0.00,0.50}{##1}}}
\expandafter\def\csname PY@tok@gd\endcsname{\def\PY@tc##1{\textcolor[rgb]{0.63,0.00,0.00}{##1}}}
\expandafter\def\csname PY@tok@gi\endcsname{\def\PY@tc##1{\textcolor[rgb]{0.00,0.63,0.00}{##1}}}
\expandafter\def\csname PY@tok@gr\endcsname{\def\PY@tc##1{\textcolor[rgb]{1.00,0.00,0.00}{##1}}}
\expandafter\def\csname PY@tok@ge\endcsname{\let\PY@it=\textit}
\expandafter\def\csname PY@tok@gs\endcsname{\let\PY@bf=\textbf}
\expandafter\def\csname PY@tok@gp\endcsname{\let\PY@bf=\textbf\def\PY@tc##1{\textcolor[rgb]{0.00,0.00,0.50}{##1}}}
\expandafter\def\csname PY@tok@go\endcsname{\def\PY@tc##1{\textcolor[rgb]{0.53,0.53,0.53}{##1}}}
\expandafter\def\csname PY@tok@gt\endcsname{\def\PY@tc##1{\textcolor[rgb]{0.00,0.27,0.87}{##1}}}
\expandafter\def\csname PY@tok@err\endcsname{\def\PY@bc##1{\setlength{\fboxsep}{0pt}\fcolorbox[rgb]{1.00,0.00,0.00}{1,1,1}{\strut ##1}}}
\expandafter\def\csname PY@tok@kc\endcsname{\let\PY@bf=\textbf\def\PY@tc##1{\textcolor[rgb]{0.00,0.50,0.00}{##1}}}
\expandafter\def\csname PY@tok@kd\endcsname{\let\PY@bf=\textbf\def\PY@tc##1{\textcolor[rgb]{0.00,0.50,0.00}{##1}}}
\expandafter\def\csname PY@tok@kn\endcsname{\let\PY@bf=\textbf\def\PY@tc##1{\textcolor[rgb]{0.00,0.50,0.00}{##1}}}
\expandafter\def\csname PY@tok@kr\endcsname{\let\PY@bf=\textbf\def\PY@tc##1{\textcolor[rgb]{0.00,0.50,0.00}{##1}}}
\expandafter\def\csname PY@tok@bp\endcsname{\def\PY@tc##1{\textcolor[rgb]{0.00,0.50,0.00}{##1}}}
\expandafter\def\csname PY@tok@fm\endcsname{\def\PY@tc##1{\textcolor[rgb]{0.00,0.00,1.00}{##1}}}
\expandafter\def\csname PY@tok@vc\endcsname{\def\PY@tc##1{\textcolor[rgb]{0.10,0.09,0.49}{##1}}}
\expandafter\def\csname PY@tok@vg\endcsname{\def\PY@tc##1{\textcolor[rgb]{0.10,0.09,0.49}{##1}}}
\expandafter\def\csname PY@tok@vi\endcsname{\def\PY@tc##1{\textcolor[rgb]{0.10,0.09,0.49}{##1}}}
\expandafter\def\csname PY@tok@vm\endcsname{\def\PY@tc##1{\textcolor[rgb]{0.10,0.09,0.49}{##1}}}
\expandafter\def\csname PY@tok@sa\endcsname{\def\PY@tc##1{\textcolor[rgb]{0.73,0.13,0.13}{##1}}}
\expandafter\def\csname PY@tok@sb\endcsname{\def\PY@tc##1{\textcolor[rgb]{0.73,0.13,0.13}{##1}}}
\expandafter\def\csname PY@tok@sc\endcsname{\def\PY@tc##1{\textcolor[rgb]{0.73,0.13,0.13}{##1}}}
\expandafter\def\csname PY@tok@dl\endcsname{\def\PY@tc##1{\textcolor[rgb]{0.73,0.13,0.13}{##1}}}
\expandafter\def\csname PY@tok@s2\endcsname{\def\PY@tc##1{\textcolor[rgb]{0.73,0.13,0.13}{##1}}}
\expandafter\def\csname PY@tok@sh\endcsname{\def\PY@tc##1{\textcolor[rgb]{0.73,0.13,0.13}{##1}}}
\expandafter\def\csname PY@tok@s1\endcsname{\def\PY@tc##1{\textcolor[rgb]{0.73,0.13,0.13}{##1}}}
\expandafter\def\csname PY@tok@mb\endcsname{\def\PY@tc##1{\textcolor[rgb]{0.40,0.40,0.40}{##1}}}
\expandafter\def\csname PY@tok@mf\endcsname{\def\PY@tc##1{\textcolor[rgb]{0.40,0.40,0.40}{##1}}}
\expandafter\def\csname PY@tok@mh\endcsname{\def\PY@tc##1{\textcolor[rgb]{0.40,0.40,0.40}{##1}}}
\expandafter\def\csname PY@tok@mi\endcsname{\def\PY@tc##1{\textcolor[rgb]{0.40,0.40,0.40}{##1}}}
\expandafter\def\csname PY@tok@il\endcsname{\def\PY@tc##1{\textcolor[rgb]{0.40,0.40,0.40}{##1}}}
\expandafter\def\csname PY@tok@mo\endcsname{\def\PY@tc##1{\textcolor[rgb]{0.40,0.40,0.40}{##1}}}
\expandafter\def\csname PY@tok@ch\endcsname{\let\PY@it=\textit\def\PY@tc##1{\textcolor[rgb]{0.25,0.50,0.50}{##1}}}
\expandafter\def\csname PY@tok@cm\endcsname{\let\PY@it=\textit\def\PY@tc##1{\textcolor[rgb]{0.25,0.50,0.50}{##1}}}
\expandafter\def\csname PY@tok@cpf\endcsname{\let\PY@it=\textit\def\PY@tc##1{\textcolor[rgb]{0.25,0.50,0.50}{##1}}}
\expandafter\def\csname PY@tok@c1\endcsname{\let\PY@it=\textit\def\PY@tc##1{\textcolor[rgb]{0.25,0.50,0.50}{##1}}}
\expandafter\def\csname PY@tok@cs\endcsname{\let\PY@it=\textit\def\PY@tc##1{\textcolor[rgb]{0.25,0.50,0.50}{##1}}}

\def\PYZbs{\char`\\}
\def\PYZus{\char`\_}
\def\PYZob{\char`\{}
\def\PYZcb{\char`\}}
\def\PYZca{\char`\^}
\def\PYZam{\char`\&}
\def\PYZlt{\char`\<}
\def\PYZgt{\char`\>}
\def\PYZsh{\char`\#}
\def\PYZpc{\char`\%}
\def\PYZdl{\char`\$}
\def\PYZhy{\char`\-}
\def\PYZsq{\char`\'}
\def\PYZdq{\char`\"}
\def\PYZti{\char`\~}
% for compatibility with earlier versions
\def\PYZat{@}
\def\PYZlb{[}
\def\PYZrb{]}
\makeatother


    % Exact colors from NB
    \definecolor{incolor}{rgb}{0.0, 0.0, 0.5}
    \definecolor{outcolor}{rgb}{0.545, 0.0, 0.0}



    
    % Prevent overflowing lines due to hard-to-break entities
    \sloppy 
    % Setup hyperref package
    \hypersetup{
      breaklinks=true,  % so long urls are correctly broken across lines
      colorlinks=true,
      urlcolor=urlcolor,
      linkcolor=linkcolor,
      citecolor=citecolor,
      }
    % Slightly bigger margins than the latex defaults
    
    \geometry{verbose,tmargin=1in,bmargin=1in,lmargin=1in,rmargin=1in}
    
    

    \begin{document}
    
    
    \maketitle
    
    

    
    \hypertarget{homework-05-conjugate-gradient-descent}{%
\section{Homework 05: Conjugate Gradient
Descent}\label{homework-05-conjugate-gradient-descent}}

    In this homework, we will implement the conjugate graident descent
algorithm. While you should nearly always use an optimization routine
from a library for practical data analyiss, this exercise is useful
because it will make concepts from multivariatble calculus and linear
algebra covered in the lectrures concrete for you. Also, it brings you
up the learning curve for the implementaiton of more complex algorithms
than the ones you have been exposed to so far.

Note: The exercise assumes that we can calculate the gradient and
Hessian of the fucntion we are trying to minimize. This can be
computationally expensive or not even possible for soeme functions.
Approximate methods can then be used; we do not go into such
complexities here.

    \textbf{Conjugate gradient descent}

We want to implement the line search method

\[ 
x_{k+1} = x_k + \alpha_k p_k
\]

where \(\alpha_k\) is the step size and \(p_k\) is the search direction.

In particular, we want the search directions \(p_k\) to be
\emph{conjugate}, as this will allow us to find the minimum in \(n\)
steps for \(x \in \mathbb{R}^n\) if \(f(x)\) is a quadratic function.

The following exercises will unpack this:

\begin{itemize}
\tightlist
\item
  What quadratic functions are
\item
  What conjugate vectors are
\item
  How to find conjugate vectors by Gram-Schmidt process
\item
  How to find the step size \(\alpha_k\)
\end{itemize}

and finally wrap them all into a conjugate gradient algorithm.

    \textbf{Quadratic function surfaces}

Recall that our objective is to minimize a scalar valued function which
maps \(\mathbb{R}^n \mapsto \mathbb{R}\), for example, a log likelihoood
function (for MLE) or unnormalized posterior distribution (for MAP).
Geometrically, we are tring to find the value of the lowest point of
some surface. The conjugate gradient algorihtm assumes that the surface
can be approximated by the quadratic expression (say, by using a Taylor
series expansion about \(x\))

\[
f(x) = \frac{1}{2}x^TAx - b^Tx + c
\]

and that

\[
\nabla f = Ax - b = 0
\]

at the minimum (if A is positive definite). Note that \(A\) is a matrix,
\(b\) is a vector, and \(c\) is a scalar. Also, note that the matrix
\(A\) is the Hessian of the quadratic function.For simplicity, we'll
work in \(\mathbb{R}^2\) so we can visualize the surface, so that \(x\)
is a 2-vector.

Note: A \textbf{form} is a polynomial function where every term has the
same degree - for example, \(x^2 + 2xy + y^2\) is a quadratic form,
whcih can be rewritten as \[
\begin{pmatrix}
x & y
\end{pmatrix}
\begin{pmatrix}
    1 & 1\\
    1 & 1
\end{pmatrix}
\begin{pmatrix}
x \\
y
\end{pmatrix}
\]

That is, \(x^TAx\) is a quadratic form.

    \begin{Verbatim}[commandchars=\\\{\}]
{\color{incolor}In [{\color{incolor}2}]:} \PY{o}{\PYZpc{}}\PY{k}{matplotlib} inline
        \PY{k+kn}{import} \PY{n+nn}{matplotlib}\PY{n+nn}{.}\PY{n+nn}{pyplot} \PY{k}{as} \PY{n+nn}{plt}
        \PY{k+kn}{import} \PY{n+nn}{numpy} \PY{k}{as} \PY{n+nn}{np}
        \PY{n}{np}\PY{o}{.}\PY{n}{set\PYZus{}printoptions}\PY{p}{(}\PY{n}{precision}\PY{o}{=}\PY{l+m+mi}{2}\PY{p}{)}
\end{Verbatim}


    \textbf{Exercise 1 (20 points)}

We will work with function \(f_1\)

\[
f1(x) = \frac{1}{2} x^T \pmatrix{1 & 0 \\ 0 & 1}x
\]

and function \(f_2\)

\[
f2(x) = \frac{1}{2} x^T \pmatrix{1 & 0 \\ 0 & 3}x
\]

\begin{itemize}
\tightlist
\item
  Plot the labeled contours of the quadratic functions
\item
  Use a \texttt{streamplot} to show the gradient vector field of the
  above quadratic functions.
\end{itemize}

    \begin{Verbatim}[commandchars=\\\{\}]
{\color{incolor}In [{\color{incolor}3}]:} \PY{k}{def} \PY{n+nf}{f1}\PY{p}{(}\PY{n}{x}\PY{p}{)}\PY{p}{:}
            \PY{k}{return} \PY{p}{(}\PY{n}{x}\PY{p}{[}\PY{l+m+mi}{0}\PY{p}{]}\PY{o}{*}\PY{o}{*}\PY{l+m+mi}{2} \PY{o}{+} \PY{n}{x}\PY{p}{[}\PY{l+m+mi}{1}\PY{p}{]}\PY{o}{*}\PY{o}{*}\PY{l+m+mi}{2}\PY{p}{)}\PY{o}{/}\PY{l+m+mi}{2}
        \PY{k}{def} \PY{n+nf}{f2}\PY{p}{(}\PY{n}{x}\PY{p}{)}\PY{p}{:}
            \PY{k}{return} \PY{p}{(}\PY{n}{x}\PY{p}{[}\PY{l+m+mi}{0}\PY{p}{]}\PY{o}{*}\PY{o}{*}\PY{l+m+mi}{2} \PY{o}{+} \PY{l+m+mi}{3}\PY{o}{*}\PY{n}{x}\PY{p}{[}\PY{l+m+mi}{1}\PY{p}{]}\PY{o}{*}\PY{o}{*}\PY{l+m+mi}{2}\PY{p}{)}\PY{o}{/}\PY{l+m+mi}{2} 
        \PY{k}{def} \PY{n+nf}{grad1}\PY{p}{(}\PY{n}{x}\PY{p}{)}\PY{p}{:}
            \PY{k}{return} \PY{n}{np}\PY{o}{.}\PY{n}{array}\PY{p}{(}\PY{p}{[}\PY{n}{x}\PY{p}{[}\PY{l+m+mi}{0}\PY{p}{]}\PY{p}{,} \PY{n}{x}\PY{p}{[}\PY{l+m+mi}{1}\PY{p}{]}\PY{p}{]}\PY{p}{)}\PY{o}{.}\PY{n}{reshape}\PY{p}{(}\PY{p}{[}\PY{o}{\PYZhy{}}\PY{l+m+mi}{1}\PY{p}{,}\PY{l+m+mi}{1}\PY{p}{]}\PY{p}{)}
        \PY{k}{def} \PY{n+nf}{grad2}\PY{p}{(}\PY{n}{x}\PY{p}{)}\PY{p}{:}
            \PY{k}{return} \PY{n}{np}\PY{o}{.}\PY{n}{array}\PY{p}{(}\PY{p}{[}\PY{n}{x}\PY{p}{[}\PY{l+m+mi}{0}\PY{p}{]}\PY{p}{,} \PY{l+m+mi}{3}\PY{o}{*}\PY{n}{x}\PY{p}{[}\PY{l+m+mi}{1}\PY{p}{]}\PY{p}{]}\PY{p}{)}\PY{o}{.}\PY{n}{reshape}\PY{p}{(}\PY{p}{[}\PY{o}{\PYZhy{}}\PY{l+m+mi}{1}\PY{p}{,}\PY{l+m+mi}{1}\PY{p}{]}\PY{p}{)}
\end{Verbatim}


    \begin{Verbatim}[commandchars=\\\{\}]
{\color{incolor}In [{\color{incolor}4}]:} \PY{n}{x} \PY{o}{=} \PY{n}{np}\PY{o}{.}\PY{n}{linspace}\PY{p}{(}\PY{o}{\PYZhy{}}\PY{l+m+mi}{8}\PY{p}{,}\PY{l+m+mi}{8}\PY{p}{,}\PY{l+m+mi}{100}\PY{p}{)}
        \PY{n}{y} \PY{o}{=} \PY{n}{np}\PY{o}{.}\PY{n}{linspace}\PY{p}{(}\PY{o}{\PYZhy{}}\PY{l+m+mi}{8}\PY{p}{,}\PY{l+m+mi}{8}\PY{p}{,}\PY{l+m+mi}{100}\PY{p}{)}
        \PY{n}{X}\PY{p}{,} \PY{n}{Y} \PY{o}{=} \PY{n}{np}\PY{o}{.}\PY{n}{meshgrid}\PY{p}{(}\PY{n}{x}\PY{p}{,}\PY{n}{y}\PY{p}{)}
        \PY{n}{Z1} \PY{o}{=} \PY{n}{f1}\PY{p}{(}\PY{n}{np}\PY{o}{.}\PY{n}{vstack}\PY{p}{(}\PY{p}{[}\PY{n}{X}\PY{o}{.}\PY{n}{ravel}\PY{p}{(}\PY{p}{)}\PY{p}{,} \PY{n}{Y}\PY{o}{.}\PY{n}{ravel}\PY{p}{(}\PY{p}{)}\PY{p}{]}\PY{p}{)}\PY{p}{)}\PY{o}{.}\PY{n}{reshape}\PY{p}{(}\PY{l+m+mi}{100}\PY{p}{,}\PY{l+m+mi}{100}\PY{p}{)}
        \PY{n}{U1} \PY{o}{=} \PY{n}{X}
        \PY{n}{V1} \PY{o}{=} \PY{n}{Y}
        \PY{n}{Z2} \PY{o}{=} \PY{n}{f2}\PY{p}{(}\PY{n}{np}\PY{o}{.}\PY{n}{vstack}\PY{p}{(}\PY{p}{[}\PY{n}{X}\PY{o}{.}\PY{n}{ravel}\PY{p}{(}\PY{p}{)}\PY{p}{,} \PY{n}{Y}\PY{o}{.}\PY{n}{ravel}\PY{p}{(}\PY{p}{)}\PY{p}{]}\PY{p}{)}\PY{p}{)}\PY{o}{.}\PY{n}{reshape}\PY{p}{(}\PY{l+m+mi}{100}\PY{p}{,}\PY{l+m+mi}{100}\PY{p}{)}
        \PY{n}{U2} \PY{o}{=} \PY{n}{X}
        \PY{n}{V2} \PY{o}{=} \PY{l+m+mi}{3}\PY{o}{*}\PY{n}{Y}
\end{Verbatim}


    \begin{Verbatim}[commandchars=\\\{\}]
{\color{incolor}In [{\color{incolor}5}]:} \PY{n}{fig}\PY{p}{,} \PY{n}{ax} \PY{o}{=} \PY{n}{plt}\PY{o}{.}\PY{n}{subplots}\PY{p}{(}\PY{p}{)}
        \PY{n}{CS} \PY{o}{=} \PY{n}{ax}\PY{o}{.}\PY{n}{contour}\PY{p}{(}\PY{n}{X}\PY{p}{,} \PY{n}{Y}\PY{p}{,} \PY{n}{Z1}\PY{p}{)}
        \PY{n}{ax}\PY{o}{.}\PY{n}{clabel}\PY{p}{(}\PY{n}{CS}\PY{p}{,} \PY{n}{CS}\PY{o}{.}\PY{n}{levels}\PY{p}{,} \PY{n}{inline}\PY{o}{=}\PY{k+kc}{True}\PY{p}{,} \PY{n}{fontsize}\PY{o}{=}\PY{l+m+mi}{10}\PY{p}{)}
        \PY{n}{ax}\PY{o}{.}\PY{n}{axis}\PY{p}{(}\PY{l+s+s1}{\PYZsq{}}\PY{l+s+s1}{square}\PY{l+s+s1}{\PYZsq{}}\PY{p}{)}
        \PY{n}{plt}\PY{o}{.}\PY{n}{title}\PY{p}{(}\PY{l+s+s1}{\PYZsq{}}\PY{l+s+s1}{Contour of f1}\PY{l+s+s1}{\PYZsq{}}\PY{p}{)}
\end{Verbatim}


\begin{Verbatim}[commandchars=\\\{\}]
{\color{outcolor}Out[{\color{outcolor}5}]:} Text(0.5, 1.0, 'Contour of f1')
\end{Verbatim}
            
    \begin{center}
    \adjustimage{max size={0.9\linewidth}{0.9\paperheight}}{output_8_1.png}
    \end{center}
    { \hspace*{\fill} \\}
    
    \begin{Verbatim}[commandchars=\\\{\}]
{\color{incolor}In [{\color{incolor}6}]:} \PY{n}{plt}\PY{o}{.}\PY{n}{streamplot}\PY{p}{(}\PY{n}{X}\PY{p}{,}\PY{n}{Y}\PY{p}{,}\PY{n}{U1}\PY{p}{,}\PY{n}{V1}\PY{p}{,} \PY{n}{color} \PY{o}{=} \PY{n}{U1}\PY{p}{)}
        \PY{n}{plt}\PY{o}{.}\PY{n}{title}\PY{p}{(}\PY{l+s+s1}{\PYZsq{}}\PY{l+s+s1}{stream plot of f1 gradient fields}\PY{l+s+s1}{\PYZsq{}}\PY{p}{)}
        \PY{n}{plt}\PY{o}{.}\PY{n}{axis}\PY{p}{(}\PY{l+s+s1}{\PYZsq{}}\PY{l+s+s1}{square}\PY{l+s+s1}{\PYZsq{}}\PY{p}{)}
\end{Verbatim}


\begin{Verbatim}[commandchars=\\\{\}]
{\color{outcolor}Out[{\color{outcolor}6}]:} (-7.999999999999999, 8.000000000000014, -7.999999999999999, 8.000000000000014)
\end{Verbatim}
            
    \begin{center}
    \adjustimage{max size={0.9\linewidth}{0.9\paperheight}}{output_9_1.png}
    \end{center}
    { \hspace*{\fill} \\}
    
    \begin{Verbatim}[commandchars=\\\{\}]
{\color{incolor}In [{\color{incolor}7}]:} \PY{n}{fig}\PY{p}{,} \PY{n}{ax} \PY{o}{=} \PY{n}{plt}\PY{o}{.}\PY{n}{subplots}\PY{p}{(}\PY{p}{)}
        \PY{n}{CS} \PY{o}{=} \PY{n}{ax}\PY{o}{.}\PY{n}{contour}\PY{p}{(}\PY{n}{X}\PY{p}{,} \PY{n}{Y}\PY{p}{,} \PY{n}{Z2}\PY{p}{)}
        \PY{n}{ax}\PY{o}{.}\PY{n}{clabel}\PY{p}{(}\PY{n}{CS}\PY{p}{,} \PY{n}{CS}\PY{o}{.}\PY{n}{levels}\PY{p}{,} \PY{n}{inline}\PY{o}{=}\PY{k+kc}{True}\PY{p}{,} \PY{n}{fontsize}\PY{o}{=}\PY{l+m+mi}{10}\PY{p}{)}
        \PY{n}{ax}\PY{o}{.}\PY{n}{axis}\PY{p}{(}\PY{l+s+s1}{\PYZsq{}}\PY{l+s+s1}{square}\PY{l+s+s1}{\PYZsq{}}\PY{p}{)}
        \PY{n}{plt}\PY{o}{.}\PY{n}{title}\PY{p}{(}\PY{l+s+s1}{\PYZsq{}}\PY{l+s+s1}{Contour of f2}\PY{l+s+s1}{\PYZsq{}}\PY{p}{)}
\end{Verbatim}


\begin{Verbatim}[commandchars=\\\{\}]
{\color{outcolor}Out[{\color{outcolor}7}]:} Text(0.5, 1.0, 'Contour of f2')
\end{Verbatim}
            
    \begin{center}
    \adjustimage{max size={0.9\linewidth}{0.9\paperheight}}{output_10_1.png}
    \end{center}
    { \hspace*{\fill} \\}
    
    \begin{Verbatim}[commandchars=\\\{\}]
{\color{incolor}In [{\color{incolor}8}]:} \PY{n}{plt}\PY{o}{.}\PY{n}{streamplot}\PY{p}{(}\PY{n}{X}\PY{p}{,}\PY{n}{Y}\PY{p}{,}\PY{n}{U2}\PY{p}{,}\PY{n}{V2}\PY{p}{,} \PY{n}{color} \PY{o}{=} \PY{n}{U2}\PY{p}{)}
        \PY{n}{plt}\PY{o}{.}\PY{n}{title}\PY{p}{(}\PY{l+s+s1}{\PYZsq{}}\PY{l+s+s1}{stream plot of f2 gradient fields}\PY{l+s+s1}{\PYZsq{}}\PY{p}{)}
        \PY{n}{plt}\PY{o}{.}\PY{n}{axis}\PY{p}{(}\PY{l+s+s1}{\PYZsq{}}\PY{l+s+s1}{square}\PY{l+s+s1}{\PYZsq{}}\PY{p}{)}
\end{Verbatim}


\begin{Verbatim}[commandchars=\\\{\}]
{\color{outcolor}Out[{\color{outcolor}8}]:} (-7.999999999999999, 8.000000000000014, -7.999999999999999, 8.000000000000014)
\end{Verbatim}
            
    \begin{center}
    \adjustimage{max size={0.9\linewidth}{0.9\paperheight}}{output_11_1.png}
    \end{center}
    { \hspace*{\fill} \\}
    
    \textbf{Gram-Schmidt}

The way to numerically find conjugate vectors is to use the Gram-Schmidt
process. Here, instead of the usual projection

\[
\text{proj}_u(v) = \frac{u \cdot v}{u \cdot u} \, u
\]

we use the generalized projection \[
\text{proj}_u(v) = \frac{uA^Tv}{uA^Tu} \, u
\]

    \textbf{Exercise 2 (30 points)}

The vectors \(u\) and \(v\) are orthogonal i.e. \(u^Tv = 0\) and
conjugate with respect to \(A\) if \(u^TAv = 0\). The geometric
intuition for conjugate vecrors \(u\) and \(v\) is that \(u\) and \(v\)
would be orthogonal if we stretched the contour plots so that it became
isotropic (same in all directions, just like when A=𝟙). Write a
Gram-Schmidt function to find orthogonal and conjuate vectors with the
following signature

\begin{Shaded}
\begin{Highlighting}[]
\KeywordTok{def}\NormalTok{ gram_schmidt(U, inner):}
    \CommentTok{"""Return an orthogonal matrix.}
\CommentTok{    }
\CommentTok{    U is a matrix of (column) vecotrs.}
\CommentTok{    inner is a function that calculates the inner product.}
\CommentTok{    }
\CommentTok{    Returns an orthogonal matrix of the same shape as U.}
\CommentTok{    """}
\end{Highlighting}
\end{Shaded}

Use this function and the appropiate inner product to plot

\begin{itemize}
\tightlist
\item
  An orhtogonal set of basis vectors for \(f_1\)
\item
  A conjugate set of basic vectors for \(f_2\)
\end{itemize}

where the first basis vector is to parallel to \(\pmatrix{1 \\ 1}\).

    \begin{Verbatim}[commandchars=\\\{\}]
{\color{incolor}In [{\color{incolor}9}]:} \PY{k}{def} \PY{n+nf}{inner}\PY{p}{(}\PY{n}{u}\PY{p}{,}\PY{n}{v}\PY{p}{,} \PY{n}{A} \PY{o}{=} \PY{k+kc}{None}\PY{p}{)}\PY{p}{:}
            \PY{k}{if} \PY{n}{A} \PY{o+ow}{is} \PY{o+ow}{not} \PY{k+kc}{None}\PY{p}{:}
                \PY{k}{return} \PY{p}{(}\PY{p}{(}\PY{n}{u}\PY{n+nd}{@A}\PY{o}{.}\PY{n}{T}\PY{n+nd}{@v}\PY{p}{)}\PY{o}{/}\PY{p}{(}\PY{n}{u}\PY{n+nd}{@A}\PY{o}{.}\PY{n}{T}\PY{n+nd}{@u}\PY{p}{)}\PY{p}{)}\PY{o}{*}\PY{n}{u}
            \PY{k}{else}\PY{p}{:}
                \PY{k}{return} \PY{p}{(}\PY{p}{(}\PY{n}{u}\PY{n+nd}{@v}\PY{p}{)}\PY{o}{/}\PY{p}{(}\PY{n}{u}\PY{n+nd}{@u}\PY{p}{)}\PY{p}{)}\PY{o}{*}\PY{n}{u}
\end{Verbatim}


    \begin{Verbatim}[commandchars=\\\{\}]
{\color{incolor}In [{\color{incolor}10}]:} \PY{k}{def} \PY{n+nf}{gram\PYZus{}schmidt}\PY{p}{(}\PY{n}{U}\PY{p}{,} \PY{n}{inner}\PY{p}{,} \PY{n}{A} \PY{o}{=} \PY{k+kc}{None}\PY{p}{)}\PY{p}{:}
             
             \PY{l+s+sd}{\PYZdq{}\PYZdq{}\PYZdq{}Return an orthogonal matrix.}
         
         \PY{l+s+sd}{    U is a matrix of (column) vecotrs.}
         \PY{l+s+sd}{    inner is a function that calculates the inner product.}
         \PY{l+s+sd}{    }
         \PY{l+s+sd}{    Returns an orthogonal matrix of the same shape as U.}
         \PY{l+s+sd}{    \PYZdq{}\PYZdq{}\PYZdq{}}
             \PY{n}{Q} \PY{o}{=} \PY{n}{U}\PY{o}{.}\PY{n}{copy}\PY{p}{(}\PY{p}{)}
             \PY{n}{n} \PY{o}{=} \PY{n+nb}{len}\PY{p}{(}\PY{n}{Q}\PY{p}{)}
             \PY{k}{for} \PY{n}{i} \PY{o+ow}{in} \PY{n+nb}{range}\PY{p}{(}\PY{l+m+mi}{1}\PY{p}{,}\PY{n}{n}\PY{p}{)}\PY{p}{:}
                 \PY{n}{qi} \PY{o}{=} \PY{n}{Q}\PY{p}{[}\PY{p}{:}\PY{p}{,}\PY{n}{i}\PY{p}{]}
                 \PY{n}{proj} \PY{o}{=} \PY{p}{[}\PY{n}{inner}\PY{p}{(}\PY{n}{Q}\PY{p}{[}\PY{p}{:}\PY{p}{,}\PY{n}{j}\PY{p}{]}\PY{p}{,}\PY{n}{qi}\PY{p}{,}\PY{n}{A}\PY{p}{)} \PY{k}{for} \PY{n}{j} \PY{o+ow}{in} \PY{n+nb}{range}\PY{p}{(}\PY{n}{i}\PY{p}{)}\PY{p}{]}
                 \PY{n}{qi} \PY{o}{=} \PY{n}{qi} \PY{o}{\PYZhy{}} \PY{n+nb}{sum}\PY{p}{(}\PY{n}{proj}\PY{p}{)}
                 \PY{n}{Q}\PY{p}{[}\PY{p}{:}\PY{p}{,}\PY{n}{i}\PY{p}{]} \PY{o}{=} \PY{n}{qi}
             \PY{k}{for} \PY{n}{i} \PY{o+ow}{in} \PY{n+nb}{range}\PY{p}{(}\PY{n}{n}\PY{p}{)}\PY{p}{:}
                 \PY{n}{Q}\PY{p}{[}\PY{p}{:}\PY{p}{,}\PY{n}{i}\PY{p}{]} \PY{o}{=} \PY{n}{Q}\PY{p}{[}\PY{p}{:}\PY{p}{,}\PY{n}{i}\PY{p}{]}\PY{o}{/}\PY{n}{np}\PY{o}{.}\PY{n}{linalg}\PY{o}{.}\PY{n}{norm}\PY{p}{(}\PY{n}{Q}\PY{p}{[}\PY{p}{:}\PY{p}{,}\PY{n}{i}\PY{p}{]}\PY{p}{)}
             \PY{k}{return} \PY{n}{Q}
\end{Verbatim}


    \begin{Verbatim}[commandchars=\\\{\}]
{\color{incolor}In [{\color{incolor}11}]:} \PY{n}{x0} \PY{o}{=} \PY{n}{np}\PY{o}{.}\PY{n}{array}\PY{p}{(}\PY{p}{[}\PY{l+m+mi}{1}\PY{p}{,}\PY{l+m+mi}{1}\PY{p}{]}\PY{p}{)}
         \PY{n}{U} \PY{o}{=} \PY{n}{np}\PY{o}{.}\PY{n}{c\PYZus{}}\PY{p}{[}\PY{n}{x0}\PY{p}{,} \PY{n}{np}\PY{o}{.}\PY{n}{random}\PY{o}{.}\PY{n}{rand}\PY{p}{(}\PY{l+m+mi}{2}\PY{p}{)}\PY{p}{]}
         \PY{n}{P1} \PY{o}{=} \PY{n}{gram\PYZus{}schmidt}\PY{p}{(}\PY{n}{U}\PY{p}{,}\PY{n}{inner}\PY{p}{)}
\end{Verbatim}


    \begin{Verbatim}[commandchars=\\\{\}]
{\color{incolor}In [{\color{incolor}14}]:} \PY{n}{A2} \PY{o}{=} \PY{n}{np}\PY{o}{.}\PY{n}{diag}\PY{p}{(}\PY{p}{[}\PY{l+m+mi}{1}\PY{p}{,}\PY{l+m+mi}{3}\PY{p}{]}\PY{p}{)}
         \PY{n}{U} \PY{o}{=} \PY{n}{np}\PY{o}{.}\PY{n}{c\PYZus{}}\PY{p}{[}\PY{n}{x0}\PY{p}{,} \PY{n}{np}\PY{o}{.}\PY{n}{random}\PY{o}{.}\PY{n}{rand}\PY{p}{(}\PY{l+m+mi}{2}\PY{p}{)}\PY{p}{]}
         \PY{n}{P2} \PY{o}{=} \PY{n}{gram\PYZus{}schmidt}\PY{p}{(}\PY{n}{U}\PY{p}{,}\PY{n}{inner}\PY{p}{,}\PY{n}{A2}\PY{p}{)}
\end{Verbatim}


    \begin{Verbatim}[commandchars=\\\{\}]
{\color{incolor}In [{\color{incolor}13}]:} \PY{n}{plt}\PY{o}{.}\PY{n}{plot}\PY{p}{(}\PY{n}{P1}\PY{p}{)}
         \PY{n}{plt}\PY{o}{.}\PY{n}{title}\PY{p}{(}\PY{l+s+s2}{\PYZdq{}}\PY{l+s+s2}{plot of f1}\PY{l+s+s2}{\PYZsq{}}\PY{l+s+s2}{s orthogonal basis}\PY{l+s+s2}{\PYZdq{}}\PY{p}{)}
\end{Verbatim}


\begin{Verbatim}[commandchars=\\\{\}]
{\color{outcolor}Out[{\color{outcolor}13}]:} Text(0.5, 1.0, "plot of f1's orthogonal basis")
\end{Verbatim}
            
    \begin{center}
    \adjustimage{max size={0.9\linewidth}{0.9\paperheight}}{output_18_1.png}
    \end{center}
    { \hspace*{\fill} \\}
    
    \begin{Verbatim}[commandchars=\\\{\}]
{\color{incolor}In [{\color{incolor}15}]:} \PY{n}{fig}\PY{p}{,} \PY{n}{ax} \PY{o}{=} \PY{n}{plt}\PY{o}{.}\PY{n}{subplots}\PY{p}{(}\PY{p}{)}
         \PY{n}{CS} \PY{o}{=} \PY{n}{ax}\PY{o}{.}\PY{n}{contour}\PY{p}{(}\PY{n}{X}\PY{p}{,} \PY{n}{Y}\PY{p}{,} \PY{n}{Z1}\PY{p}{)}
         \PY{n}{ax}\PY{o}{.}\PY{n}{clabel}\PY{p}{(}\PY{n}{CS}\PY{p}{,} \PY{n}{CS}\PY{o}{.}\PY{n}{levels}\PY{p}{,} \PY{n}{inline}\PY{o}{=}\PY{k+kc}{True}\PY{p}{,} \PY{n}{fontsize}\PY{o}{=}\PY{l+m+mi}{10}\PY{p}{)}
         \PY{n}{ax}\PY{o}{.}\PY{n}{axis}\PY{p}{(}\PY{l+s+s1}{\PYZsq{}}\PY{l+s+s1}{square}\PY{l+s+s1}{\PYZsq{}}\PY{p}{)}
         \PY{n}{plt}\PY{o}{.}\PY{n}{title}\PY{p}{(}\PY{l+s+s1}{\PYZsq{}}\PY{l+s+s1}{Contour of f1 and orthgonal set of basis}\PY{l+s+s1}{\PYZsq{}}\PY{p}{)}
         \PY{n}{ax}\PY{o}{.}\PY{n}{plot}\PY{p}{(}\PY{n}{P1}\PY{p}{)}
\end{Verbatim}


\begin{Verbatim}[commandchars=\\\{\}]
{\color{outcolor}Out[{\color{outcolor}15}]:} [<matplotlib.lines.Line2D at 0x7fe49d7bff98>,
          <matplotlib.lines.Line2D at 0x7fe49d7cfe48>]
\end{Verbatim}
            
    \begin{center}
    \adjustimage{max size={0.9\linewidth}{0.9\paperheight}}{output_19_1.png}
    \end{center}
    { \hspace*{\fill} \\}
    
    \begin{Verbatim}[commandchars=\\\{\}]
{\color{incolor}In [{\color{incolor}16}]:} \PY{n}{plt}\PY{o}{.}\PY{n}{plot}\PY{p}{(}\PY{n}{P2}\PY{p}{)}
         \PY{n}{plt}\PY{o}{.}\PY{n}{title}\PY{p}{(}\PY{l+s+s2}{\PYZdq{}}\PY{l+s+s2}{plot of f2}\PY{l+s+s2}{\PYZsq{}}\PY{l+s+s2}{s conjugate basis}\PY{l+s+s2}{\PYZdq{}}\PY{p}{)}
\end{Verbatim}


\begin{Verbatim}[commandchars=\\\{\}]
{\color{outcolor}Out[{\color{outcolor}16}]:} Text(0.5, 1.0, "plot of f2's conjugate basis")
\end{Verbatim}
            
    \begin{center}
    \adjustimage{max size={0.9\linewidth}{0.9\paperheight}}{output_20_1.png}
    \end{center}
    { \hspace*{\fill} \\}
    
    \begin{Verbatim}[commandchars=\\\{\}]
{\color{incolor}In [{\color{incolor}17}]:} \PY{n}{fig}\PY{p}{,} \PY{n}{ax} \PY{o}{=} \PY{n}{plt}\PY{o}{.}\PY{n}{subplots}\PY{p}{(}\PY{p}{)}
         \PY{n}{CS} \PY{o}{=} \PY{n}{ax}\PY{o}{.}\PY{n}{contour}\PY{p}{(}\PY{n}{X}\PY{p}{,} \PY{n}{Y}\PY{p}{,} \PY{n}{Z2}\PY{p}{)}
         \PY{n}{ax}\PY{o}{.}\PY{n}{clabel}\PY{p}{(}\PY{n}{CS}\PY{p}{,} \PY{n}{CS}\PY{o}{.}\PY{n}{levels}\PY{p}{,} \PY{n}{inline}\PY{o}{=}\PY{k+kc}{True}\PY{p}{,} \PY{n}{fontsize}\PY{o}{=}\PY{l+m+mi}{10}\PY{p}{)}
         \PY{n}{ax}\PY{o}{.}\PY{n}{axis}\PY{p}{(}\PY{l+s+s1}{\PYZsq{}}\PY{l+s+s1}{square}\PY{l+s+s1}{\PYZsq{}}\PY{p}{)}
         \PY{n}{plt}\PY{o}{.}\PY{n}{title}\PY{p}{(}\PY{l+s+s1}{\PYZsq{}}\PY{l+s+s1}{Contour of f2 and conjugate set of basis}\PY{l+s+s1}{\PYZsq{}}\PY{p}{)}
         \PY{n}{ax}\PY{o}{.}\PY{n}{plot}\PY{p}{(}\PY{n}{P2}\PY{p}{)}
\end{Verbatim}


\begin{Verbatim}[commandchars=\\\{\}]
{\color{outcolor}Out[{\color{outcolor}17}]:} [<matplotlib.lines.Line2D at 0x7fe49d6cb470>,
          <matplotlib.lines.Line2D at 0x7fe49d6dbbe0>]
\end{Verbatim}
            
    \begin{center}
    \adjustimage{max size={0.9\linewidth}{0.9\paperheight}}{output_21_1.png}
    \end{center}
    { \hspace*{\fill} \\}
    
    \textbf{Exercise 3 (20 points)}

We now need to find the ``step size'' \(\alpha\) to take in the
direction of the search vector \(p\). We can get a quadratic
approximation to a general nonliner function \(f\) by taking the Taylor
series in the driection of \(p\)

\[
f(x + \alpha p) = f(x) + \alpha [f'(x)]^T p + \frac{\alpha^2}{2} p^T f''(x) p
\]

Find the derivative with respect to \(\alpha\) and use this to find the
optimal value for \(\alpha\) with respect to the quadratic
approcimaiton.

\begin{itemize}
\tightlist
\item
  Write the derivation of the optimal step size \(\alpha\) using LaTeX
  in a Markdown cell.
\item
  Write a funciton that returns \(\alpha\) for a quadratic funciton with
  the following signature
\end{itemize}

\begin{Shaded}
\begin{Highlighting}[]
\KeywordTok{def}\NormalTok{ step(x, p, A, b):}
    \CommentTok{"""Returns the optimal step size to take in line search on a quadratic.}
\CommentTok{    }
\CommentTok{    A and b are the coefficients of the quadartic expression }
\CommentTok{    $$}
\CommentTok{    f(x) = \textbackslash{}frac\{1\}\{2\}x^TAx - b^Tx + c}
\CommentTok{    $$}
\CommentTok{    p is the search direction}
\CommentTok{    x is the current location}
\CommentTok{    """}
\end{Highlighting}
\end{Shaded}

    \[\frac{df(x+\alpha p)}{d\alpha} = [f'(x)]^Tp + \alpha p^Tf''(x)p = 0 \\
\alpha = -\frac{[f'(x)]^Tp}{p^Tf''(X)p} =  - \frac{\nabla f^T p}{p^T A p} 
= \frac{(b - Ax)^T p}{p^T A p}
\]

    \[
(b_1 - A_1x)^T = [f1'(x)]^T = [x, y], \\
(b_2 - A_2x)^T =[f2'(x)]^T = [x,3y]
\] \[A_1 = f1''(x) = \begin{bmatrix}1 &0\\ 0&1\end{bmatrix}, \\
A_2 = f2''(x) = \begin{bmatrix}1 &0\\ 0&3\end{bmatrix}\]

    \[\alpha_1 = \frac{[x,y]p}{p^Tp}, \alpha_2 = \frac{[x,3y]p}{p^Tf2''(x)p}\]

    \begin{Verbatim}[commandchars=\\\{\}]
{\color{incolor}In [{\color{incolor}18}]:} \PY{k}{def} \PY{n+nf}{step}\PY{p}{(}\PY{n}{x}\PY{p}{,} \PY{n}{p}\PY{p}{,} \PY{n}{A}\PY{p}{,} \PY{n}{b}\PY{p}{)}\PY{p}{:}
             \PY{l+s+sd}{\PYZdq{}\PYZdq{}\PYZdq{}Returns the optimal step size to take in line search on a quadratic.}
         \PY{l+s+sd}{    }
         \PY{l+s+sd}{    A and b are the coefficients of the quadartic expression }
         \PY{l+s+sd}{    \PYZdl{}\PYZdl{}}
         \PY{l+s+sd}{    f(x) = \PYZbs{}frac\PYZob{}1\PYZcb{}\PYZob{}2\PYZcb{}x\PYZca{}TAx \PYZhy{} b\PYZca{}Tx + c}
         \PY{l+s+sd}{    \PYZdl{}\PYZdl{}}
         \PY{l+s+sd}{    p is the search direction}
         \PY{l+s+sd}{    x is the current location}
         \PY{l+s+sd}{    \PYZdq{}\PYZdq{}\PYZdq{}}
             \PY{k}{return} \PY{p}{(}\PY{p}{(}\PY{n}{b}\PY{o}{\PYZhy{}}\PY{n}{A}\PY{n+nd}{@x}\PY{p}{)}\PY{o}{.}\PY{n}{T}\PY{n+nd}{@p}\PY{p}{)}\PY{o}{/}\PY{p}{(}\PY{n}{p}\PY{o}{.}\PY{n}{T}\PY{n+nd}{@A}\PY{n+nd}{@p}\PY{p}{)}
\end{Verbatim}


    \begin{Verbatim}[commandchars=\\\{\}]
{\color{incolor}In [{\color{incolor}19}]:} \PY{n}{x0} \PY{o}{=} \PY{p}{[}\PY{l+m+mi}{6}\PY{p}{,}\PY{l+m+mi}{7}\PY{p}{]}
         \PY{n}{A1} \PY{o}{=} \PY{n}{np}\PY{o}{.}\PY{n}{eye}\PY{p}{(}\PY{l+m+mi}{2}\PY{p}{)}
         \PY{n}{A2} \PY{o}{=} \PY{n}{np}\PY{o}{.}\PY{n}{diag}\PY{p}{(}\PY{p}{[}\PY{l+m+mi}{1}\PY{p}{,}\PY{l+m+mi}{3}\PY{p}{]}\PY{p}{)}
\end{Verbatim}


    \begin{Verbatim}[commandchars=\\\{\}]
{\color{incolor}In [{\color{incolor}20}]:} \PY{n}{step}\PY{p}{(}\PY{n}{x0}\PY{p}{,} \PY{n}{P1}\PY{p}{[}\PY{p}{:}\PY{p}{,}\PY{l+m+mi}{0}\PY{p}{]}\PY{p}{,}\PY{n}{A1}\PY{p}{,}\PY{n}{b} \PY{o}{=} \PY{l+m+mi}{0}\PY{p}{)}
\end{Verbatim}


\begin{Verbatim}[commandchars=\\\{\}]
{\color{outcolor}Out[{\color{outcolor}20}]:} -9.19238815542512
\end{Verbatim}
            
    \begin{Verbatim}[commandchars=\\\{\}]
{\color{incolor}In [{\color{incolor}21}]:} \PY{n}{step}\PY{p}{(}\PY{n}{x0}\PY{p}{,} \PY{n}{P2}\PY{p}{[}\PY{p}{:}\PY{p}{,}\PY{l+m+mi}{0}\PY{p}{]}\PY{p}{,}\PY{n}{A2}\PY{p}{,}\PY{n}{b} \PY{o}{=} \PY{l+m+mi}{0}\PY{p}{)}
\end{Verbatim}


\begin{Verbatim}[commandchars=\\\{\}]
{\color{outcolor}Out[{\color{outcolor}21}]:} -9.545941546018392
\end{Verbatim}
            
    \textbf{Line search}

We now know how to find a search direction \(p_k\) - this is a vector
that is conjugate to the previous search direction. The first search
direction is usually set to be the gradient. Next we need to find out
how far along \(p_k\) we need to travel, i.e., we need to find
\(\alpha_k\). First we take a Taylor expansion in the direction of \(p\)

\[
f(x + \alpha p) = f(x) + \alpha [f'(x)]^T p + \frac{\alpha^2}{2} p^T f''(x) p
\]

followed by finding the derivative with respect to \(\alpha\)

\[
\frac{d}{d\alpha} f(x + \alpha p) = [f'(x)]^T p + \alpha p^T f''(x) p
\]

Solvign for \(\frac{d}{d\alpha} f(x + \alpha p) = 0\), we get

\[
\alpha = - \frac{[f'(x)]^T p}{p^T f''(x) p} \\
= - \frac{\nabla f^T p}{p^T A p} \\
= \frac{(b - Ax)^T p}{p^T A p}
\]

    \textbf{Exercise 4 (30 points)}

Implement the conjugate grdient descent algorithm with the following
signature

\begin{Shaded}
\begin{Highlighting}[]
\KeywordTok{def}\NormalTok{ cg(x, A, b, c, max_iter}\OperatorTok{=}\DecValTok{100}\NormalTok{, tol}\OperatorTok{=}\FloatTok{1e-3}\NormalTok{):}
    \CommentTok{"""Conjugate gradient descent on a quadratic function surface.}
\CommentTok{    }
\CommentTok{    x is the starting position}
\CommentTok{    A, b and c are the coefficients of the quadartic expression }
\CommentTok{    $$}
\CommentTok{    f(x) = \textbackslash{}frac\{1\}\{2\}x^TAx - b^Tx + c}
\CommentTok{    $$}
\CommentTok{    max_iter is the maximum number of iterations to take}
\CommentTok{    tol is the tolerance (stop if the length of the gradient is smaller than tol)}

\CommentTok{    Returns the number of steps taken and the list of all positions visited.}
\CommentTok{    """}
\end{Highlighting}
\end{Shaded}

\begin{itemize}
\item
  Use cg to find the minimum of the funciton \(f_2\) from Exercise 1,
  starting from \(\pmatrix{6 \\ 7}\).
\item
  Plot the contour of the funciton f and the trajectory taken from the
  inital starting poitn \(x\) to the final position, inlcuding all the
  intermediate steps.
\end{itemize}

We are not particularly concerned about efficiency here, so don't worry
about JIT/AOT/C++ level optimization.

    \begin{Verbatim}[commandchars=\\\{\}]
{\color{incolor}In [{\color{incolor}22}]:} \PY{k}{def} \PY{n+nf}{cg}\PY{p}{(}\PY{n}{x}\PY{p}{,} \PY{n}{A}\PY{p}{,} \PY{n}{b}\PY{p}{,} \PY{n}{c}\PY{p}{,} \PY{n}{max\PYZus{}iter}\PY{o}{=}\PY{l+m+mi}{100}\PY{p}{,} \PY{n}{tol}\PY{o}{=}\PY{l+m+mf}{1e\PYZhy{}3}\PY{p}{)}\PY{p}{:}
             \PY{n}{orbit} \PY{o}{=} \PY{n}{np}\PY{o}{.}\PY{n}{zeros}\PY{p}{(}\PY{p}{(}\PY{n+nb}{len}\PY{p}{(}\PY{n}{x}\PY{p}{)}\PY{p}{,}\PY{n}{max\PYZus{}iter}\PY{o}{+}\PY{l+m+mi}{1}\PY{p}{)}\PY{p}{)}
             \PY{n}{p} \PY{o}{=} \PY{n}{np}\PY{o}{.}\PY{n}{zeros}\PY{p}{(}\PY{p}{(}\PY{n+nb}{len}\PY{p}{(}\PY{n}{x}\PY{p}{)}\PY{p}{,}\PY{n}{max\PYZus{}iter}\PY{o}{+}\PY{l+m+mi}{1}\PY{p}{)}\PY{p}{)}
             \PY{n}{orbit}\PY{p}{[}\PY{p}{:}\PY{p}{,}\PY{l+m+mi}{0}\PY{p}{]} \PY{o}{=} \PY{n}{x}\PY{o}{.}\PY{n}{ravel}\PY{p}{(}\PY{p}{)}
             \PY{n}{p}\PY{p}{[}\PY{p}{:}\PY{p}{,}\PY{l+m+mi}{0}\PY{p}{]} \PY{o}{=} \PY{n}{np}\PY{o}{.}\PY{n}{array}\PY{p}{(}\PY{p}{[}\PY{l+m+mi}{1}\PY{p}{,}\PY{l+m+mi}{1}\PY{p}{]}\PY{p}{)}
             \PY{n}{count} \PY{o}{=} \PY{l+m+mi}{1}
             \PY{k}{for} \PY{n}{i} \PY{o+ow}{in} \PY{n+nb}{range}\PY{p}{(}\PY{n}{max\PYZus{}iter}\PY{p}{)}\PY{p}{:}
                 \PY{n}{count} \PY{o}{+}\PY{o}{=} \PY{l+m+mi}{1}
                 \PY{n}{pi} \PY{o}{=} \PY{n}{p}\PY{p}{[}\PY{p}{:}\PY{p}{,}\PY{n}{i}\PY{p}{]}
                 \PY{n}{alpha} \PY{o}{=} \PY{n}{step}\PY{p}{(}\PY{n}{x}\PY{p}{,} \PY{n}{pi}\PY{p}{,} \PY{n}{A}\PY{p}{,} \PY{n}{b}\PY{p}{)}
                 \PY{n}{x} \PY{o}{=} \PY{n}{x} \PY{o}{+} \PY{n}{alpha}\PY{o}{*}\PY{n}{pi}
                 \PY{n}{orbit}\PY{p}{[}\PY{p}{:}\PY{p}{,}\PY{n}{i}\PY{o}{+}\PY{l+m+mi}{1}\PY{p}{]} \PY{o}{=} \PY{n}{x}\PY{o}{.}\PY{n}{ravel}\PY{p}{(}\PY{p}{)}
                 \PY{k}{if}\PY{p}{(}\PY{n}{np}\PY{o}{.}\PY{n}{linalg}\PY{o}{.}\PY{n}{norm}\PY{p}{(}\PY{n}{A}\PY{n+nd}{@x}\PY{o}{\PYZhy{}}\PY{n}{b}\PY{p}{)}\PY{o}{\PYZlt{}}\PY{n}{tol}\PY{p}{)}\PY{p}{:}
                     \PY{k}{break}
                 \PY{n}{p}\PY{p}{[}\PY{p}{:}\PY{p}{,}\PY{n}{i}\PY{o}{+}\PY{l+m+mi}{1}\PY{p}{]} \PY{o}{=} \PY{p}{(}\PY{n}{A}\PY{n+nd}{@x}\PY{o}{\PYZhy{}}\PY{n}{b}\PY{p}{)}
                 \PY{n}{p} \PY{o}{=} \PY{n}{gram\PYZus{}schmidt}\PY{p}{(}\PY{n}{p}\PY{p}{,}\PY{n}{inner}\PY{p}{,}\PY{n}{A}\PY{p}{)}
             \PY{n}{orbit} \PY{o}{=} \PY{n}{orbit}\PY{p}{[}\PY{p}{:}\PY{p}{,}\PY{p}{:}\PY{p}{(}\PY{n}{count}\PY{p}{)}\PY{p}{]}
             \PY{k}{return} \PY{n}{orbit}\PY{o}{.}\PY{n}{T}
             \PY{l+s+sd}{\PYZdq{}\PYZdq{}\PYZdq{}Conjugate gradient descent on a quadratic function surface.}
         \PY{l+s+sd}{    }
         \PY{l+s+sd}{    x is the starting position}
         \PY{l+s+sd}{    A, b and c are the coefficients of the quadartic expression }
         \PY{l+s+sd}{    \PYZdl{}\PYZdl{}}
         \PY{l+s+sd}{    f(x) = \PYZbs{}frac\PYZob{}1\PYZcb{}\PYZob{}2\PYZcb{}x\PYZca{}TAx \PYZhy{} b\PYZca{}Tx + c}
         \PY{l+s+sd}{    \PYZdl{}\PYZdl{}}
         \PY{l+s+sd}{    max\PYZus{}iter is the maximum number of iterations to take}
         \PY{l+s+sd}{    tol is the tolerance (stop if the length of the gradient is smaller than tol)}
         
         \PY{l+s+sd}{    Returns the number of steps taken and the list of all positions visited.}
         \PY{l+s+sd}{    \PYZdq{}\PYZdq{}\PYZdq{}}
\end{Verbatim}


    \begin{Verbatim}[commandchars=\\\{\}]
{\color{incolor}In [{\color{incolor}23}]:} \PY{n}{x0} \PY{o}{=} \PY{n}{np}\PY{o}{.}\PY{n}{array}\PY{p}{(}\PY{p}{[}\PY{l+m+mi}{6}\PY{p}{,}\PY{l+m+mi}{7}\PY{p}{]}\PY{p}{)}
\end{Verbatim}


    \begin{Verbatim}[commandchars=\\\{\}]
{\color{incolor}In [{\color{incolor}24}]:} \PY{n}{orbits} \PY{o}{=} \PY{n}{cg}\PY{p}{(}\PY{n}{x0}\PY{p}{,} \PY{n}{A2}\PY{p}{,} \PY{n}{b}\PY{o}{=}\PY{l+m+mi}{0}\PY{p}{,} \PY{n}{c}\PY{o}{=}\PY{l+m+mi}{0}\PY{p}{)}
\end{Verbatim}


    \begin{Verbatim}[commandchars=\\\{\}]
{\color{incolor}In [{\color{incolor}27}]:} \PY{n}{fig}\PY{p}{,} \PY{n}{ax} \PY{o}{=} \PY{n}{plt}\PY{o}{.}\PY{n}{subplots}\PY{p}{(}\PY{p}{)}
         \PY{n}{CS} \PY{o}{=} \PY{n}{ax}\PY{o}{.}\PY{n}{contour}\PY{p}{(}\PY{n}{X}\PY{p}{,} \PY{n}{Y}\PY{p}{,} \PY{n}{Z2}\PY{p}{)}
         \PY{n}{ax}\PY{o}{.}\PY{n}{clabel}\PY{p}{(}\PY{n}{CS}\PY{p}{,} \PY{n}{CS}\PY{o}{.}\PY{n}{levels}\PY{p}{,} \PY{n}{inline}\PY{o}{=}\PY{k+kc}{True}\PY{p}{,} \PY{n}{fontsize}\PY{o}{=}\PY{l+m+mi}{10}\PY{p}{)}
         \PY{n}{ax}\PY{o}{.}\PY{n}{plot}\PY{p}{(}\PY{n}{orbits}\PY{p}{[}\PY{p}{:}\PY{p}{,} \PY{l+m+mi}{0}\PY{p}{]}\PY{p}{,} \PY{n}{orbits}\PY{p}{[}\PY{p}{:}\PY{p}{,} \PY{l+m+mi}{1}\PY{p}{]}\PY{p}{,} \PY{l+s+s1}{\PYZsq{}}\PY{l+s+s1}{\PYZhy{}}\PY{l+s+s1}{\PYZsq{}}\PY{p}{)}
         \PY{n}{ax}\PY{o}{.}\PY{n}{scatter}\PY{p}{(}\PY{n}{orbits}\PY{p}{[}\PY{p}{:}\PY{p}{,} \PY{l+m+mi}{0}\PY{p}{]}\PY{p}{,} \PY{n}{orbits}\PY{p}{[}\PY{p}{:}\PY{p}{,} \PY{l+m+mi}{1}\PY{p}{]}\PY{p}{)}
         \PY{n}{ax}\PY{o}{.}\PY{n}{axis}\PY{p}{(}\PY{l+s+s1}{\PYZsq{}}\PY{l+s+s1}{square}\PY{l+s+s1}{\PYZsq{}}\PY{p}{)}
         \PY{n}{plt}\PY{o}{.}\PY{n}{title}\PY{p}{(}\PY{l+s+s1}{\PYZsq{}}\PY{l+s+s1}{Contour of f2 and conjugate gradient step}\PY{l+s+s1}{\PYZsq{}}\PY{p}{)}
\end{Verbatim}


\begin{Verbatim}[commandchars=\\\{\}]
{\color{outcolor}Out[{\color{outcolor}27}]:} Text(0.5, 1.0, 'Contour of f2 and conjugate gradient step')
\end{Verbatim}
            
    \begin{center}
    \adjustimage{max size={0.9\linewidth}{0.9\paperheight}}{output_35_1.png}
    \end{center}
    { \hspace*{\fill} \\}
    
    \begin{Verbatim}[commandchars=\\\{\}]
{\color{incolor}In [{\color{incolor}29}]:} \PY{n}{orbits} \PY{o}{=} \PY{n}{cg}\PY{p}{(}\PY{n}{x0}\PY{p}{,} \PY{n}{A1}\PY{p}{,} \PY{n}{b}\PY{o}{=}\PY{l+m+mi}{0}\PY{p}{,} \PY{n}{c}\PY{o}{=}\PY{l+m+mi}{0}\PY{p}{)}
         \PY{n}{fig}\PY{p}{,} \PY{n}{ax} \PY{o}{=} \PY{n}{plt}\PY{o}{.}\PY{n}{subplots}\PY{p}{(}\PY{p}{)}
         \PY{n}{CS} \PY{o}{=} \PY{n}{ax}\PY{o}{.}\PY{n}{contour}\PY{p}{(}\PY{n}{X}\PY{p}{,} \PY{n}{Y}\PY{p}{,} \PY{n}{Z1}\PY{p}{)}
         \PY{n}{ax}\PY{o}{.}\PY{n}{clabel}\PY{p}{(}\PY{n}{CS}\PY{p}{,} \PY{n}{CS}\PY{o}{.}\PY{n}{levels}\PY{p}{,} \PY{n}{inline}\PY{o}{=}\PY{k+kc}{True}\PY{p}{,} \PY{n}{fontsize}\PY{o}{=}\PY{l+m+mi}{10}\PY{p}{)}
         \PY{n}{ax}\PY{o}{.}\PY{n}{plot}\PY{p}{(}\PY{n}{orbits}\PY{p}{[}\PY{p}{:}\PY{p}{,} \PY{l+m+mi}{0}\PY{p}{]}\PY{p}{,} \PY{n}{orbits}\PY{p}{[}\PY{p}{:}\PY{p}{,} \PY{l+m+mi}{1}\PY{p}{]}\PY{p}{,} \PY{l+s+s1}{\PYZsq{}}\PY{l+s+s1}{\PYZhy{}}\PY{l+s+s1}{\PYZsq{}}\PY{p}{)}
         \PY{n}{ax}\PY{o}{.}\PY{n}{scatter}\PY{p}{(}\PY{n}{orbits}\PY{p}{[}\PY{p}{:}\PY{p}{,} \PY{l+m+mi}{0}\PY{p}{]}\PY{p}{,} \PY{n}{orbits}\PY{p}{[}\PY{p}{:}\PY{p}{,} \PY{l+m+mi}{1}\PY{p}{]}\PY{p}{)}
         \PY{n}{ax}\PY{o}{.}\PY{n}{axis}\PY{p}{(}\PY{l+s+s1}{\PYZsq{}}\PY{l+s+s1}{square}\PY{l+s+s1}{\PYZsq{}}\PY{p}{)}
         \PY{n}{plt}\PY{o}{.}\PY{n}{title}\PY{p}{(}\PY{l+s+s1}{\PYZsq{}}\PY{l+s+s1}{Contour of f1 and conjugate gradient step}\PY{l+s+s1}{\PYZsq{}}\PY{p}{)}
\end{Verbatim}


\begin{Verbatim}[commandchars=\\\{\}]
{\color{outcolor}Out[{\color{outcolor}29}]:} Text(0.5, 1.0, 'Contour of f1 and conjugate gradient step')
\end{Verbatim}
            
    \begin{center}
    \adjustimage{max size={0.9\linewidth}{0.9\paperheight}}{output_36_1.png}
    \end{center}
    { \hspace*{\fill} \\}
    

    % Add a bibliography block to the postdoc
    
    
    
    \end{document}
